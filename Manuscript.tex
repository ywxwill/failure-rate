\documentclass[journal,twoside,web]{ieeecolor}
\usepackage{generic}
\usepackage{cite}
\usepackage{amsmath,amssymb,amsfonts}
\usepackage{algorithmic}
\usepackage{graphicx}
\usepackage{textcomp}
\usepackage{verbatim}
\usepackage{enumerate}
\usepackage{stfloats} %而h:here,表示在此处;t:top,表示在顶部,b:bottom,表示底部,p:page,表示在本页
\usepackage{bm}%专门处理数学粗体的bm宏包
\usepackage{multirow}%专门处理表格跨列
\usepackage{makecell}%专门处理表格跨列
\usepackage{indentfirst} %首行缩进
%!!!!!!!!

\def\BibTeX{{\rm B\kern-.05em{\sc i\kern-.025em b}\kern-.08em
    T\kern-.1667em\lower.7ex\hbox{E}\kern-.125emX}}
\markboth{\journalname, VOL. XX, NO. XX, XXXX 2017}
{}


\begin{document}
\title{Failure Rate Evaluation of Electric Energy Metering Equipments Using Weighted Nonlinear Bayesian}
\author{Wei Qiu, Qiu Tang, Wenxuan Yao, Yuhong Qin
}

\maketitle

\begin{abstract}
As one of the indicators of the normal operation of Electric Energy Metering Equipment (EEME), failure rate is essential for the accurate measurement of electric energy. However, actual failure data is often affected by insufficient sample size and environmental noise.
To address this problem, this paper first proposes a Hybrid-based Outlier Detection (HOD) method. It combines the score characteristics of k-Nearest Neighbor (kNN) and is further constrained by Chauvenet criteria to prevent false detections. 
Then, the Weighted Nonlinear Bayesian (WNB) method is proposed to fuse multiple environmental stresses and failure rate using the weight generated by the HOD. 
Combining WNB and HOD, examples from three typical environmental regions show that the proposed evaluation framework has a higher prediction performance and less uncertainty. 
Compared with the classical prediction methods, our framework has profound outlier detection and failure rate prediction performance ever under small samples.
More importantly, this method is interpretable compared to some of the current methods.


\end{abstract}

\begin{IEEEkeywords}
Electric Energy Metering Equipment, failure rate, Weighted Nonlinear Bayesian, Hybrid-based Outlier Detection
\end{IEEEkeywords}

\section{Introduction}
\label{Section1}
Electric Energy Metering Equipments (EEMEs) have been widely used around the globe in industrial and civil energy metering \cite{8322199}. EEME contain various concentrators and meters, such as smart electricity meter, heat meter and gas meter \cite{7365417}. The number of EEMEs has reached 70 million and 96 million in U.S. and China, respectively at the end of 2016 \cite{8322199}. According to \cite{7063262}, the smart electricity meter numbers in 35 countries have doubled in just one year.
Hereby, the normal operation of the EEME is important for the measurement of electrical energy. The electronic components of the EEME are prone to aging  especially in typical natural environmental conditions, thus causing failure.
Unfortunately, the failure sample data in the natural environment is limited because the EEMEs are widely distributed and difficult to collect in time. Meanwhile, the samples obtained from the wild are generally more susceptible to contamination \cite{6710227}. To provide accurate recommendations for the reliability design and rotation strategy of EEMEs, effective reliability evaluation and failure rate prediction methodologies are in demand.

In recent years, many methods have been proposed for system reliability analysis. Generally, these methods can be categorised into two main types: the direct assessment and data-driven approach \cite{LIPU2018115, 7904628}. 
The direct assessment method utilizes the failure rate of each component of the electronic device to directly analyze the reliability or remaining life. For example, the equipment inspection data at different stages were mapped to a failure rate function based on exponential model \cite{1294982}, then failure rate can be obtained combined with the equipment condition. In \cite{8617681}, component-level reliability assessment model was built to predict and calculate reliability of power converter using the reliability prediction handbook. The fault tree analysis, as a classic reliability analysis method, also has many applications in power systems \cite{7384776}. However, the direct assessment methods have a common drawback, which are difficult to apply to system-level failure analysis because component failure data is required.

Data-driven approach can use failure data of the system to evaluate its reliability, which does not require the structural composition of the instrument. Data-driven approach can be further divided into deterministic methods and probabilistic methods \cite{7904628}. The deterministic methods includes Artificial Neural Networks (ANN)\cite{KUTYLOWSKA201541} and Support Vector Regression (SVR) \cite{8186602}. For example, the ANN and Weibull distribution were combined to predict bearing remaining life \cite{article1}. However, the sample size of failure rate data is far too small to generate accurate predictions in most cases, so that other methods need to be pursued. 

The probabilistic methods can effectively deal with parameter uncertain problems. Common probabilistic methods consist of degenerate model, proportional hazards model and Bayesian model \cite{LIU201839}. For example, the proportional hazards model was proposed to analysis the failure rate of electric power equipment using the health indexs \cite{7018977}. Bayesian model is a data analysis method especially suitable for small samples \cite{QIU201921}. Thus it is often used for failure analysis. Different regression analysis methods were integrated into Bayesian to fuse multiple ages and environmental variables, such as linear regression and exponential regression \cite{ 1046900, GUO2015173}. However, the multivariate relationship is difficult to accurately described by this linear superposition. Therefore, the nonlinear regression was further proposed to enhance data fitting ability \cite{FUMO2015332}.
%On the other hand, the Bayesian Bagging was proposed to improve the accuracy in small sample datasets \cite{FUSHIKI201065}. 
However, most of the data-driven approaches ignore the effect of outliers in sample data.

Since the general Bayesian is insensitive to outliers, the anomaly information will reduce the prediction effect of the failure rate.
To this end, a general latent assignment approach was proposed for modeling outliers. However, the identification of noise data is susceptible to the influence of empirical distribution \cite{ZEIGENFUSE2010352}.  Obviously, directly deleting data  may lead to information loss especially for small sample data \cite{MOGHADDASS2019561}.
%The service life of the entire meter is equal or greater than 10 years.

This study aims to improve the accuracy of failure rate estimation of EEME, our contributions are listed as follows:

\begin{enumerate}
	\item To mitigate the impact of outliers on failure rate prediction, a Hybrid-based Outlier Detection (HOD) method is proposed to detect outliers without knowing the truth labels. The k-Nearest Neighbor (kNN) is combined with Chauvenet criteria to reduce false identification of failure rate. The deviation degree score of the failure data is specified to reduce sample information loss.

	\item To establish the relationship between environmental stress and failure rate, a weighted nonlinear regression is proposed to form a Weighted Nonlinear Bayesian (WNB) for fusing multi-region stress characteristics. Particularly, reliability and different stress coefficients  can be predicted and analyzed, respectively.
	
	\item We also propose a failure rate assessment framework of EEME based on HOD and WNB. In this framework, small sample factors and unavoidable outliers in the sample can be considered simultaneously.
	
	\item A smart electricity meter dataset collected from three typical environmental areas is used to validate the proposed framework.
	Different methods and experiments are carried out to verify the prediction results. Experimental results show that our framework has a better performance compared with some other common methods.

\end{enumerate}





The rest of the paper is organized as follows: Section \ref{Section1:data} first describes the data acquisition system of the energy metering equipment. The HOD method and weight setting are introducted in Section \ref{Section3:knn}. Then, the proposed WNB model is presented in Section \ref{Section2:bayes}. Section \ref{Section4} used a actual failure sample data set to test the WNB model. Finally, Section \ref{Section5} concludes the paper. 


\section{Data acquisition system}
\label{Section1:data}
In the actual energy measurement process, the energy metering equipments are widely distributed. Therefore, a data acquisition system is used to collect and integrate data. Fig. \ref{fig1} shows the Advanced Metering Infrastructure(AMI) system of EEMEs. The power consumption information and transmission status of the EEME are first transmitted to the EEME collector through the power line. Then the base station server obtains information through the base station.

\begin{figure}
	\centerline{\includegraphics[width=0.47\textwidth]{Picture//Dataframework.png}}
	\caption{The AMI system of EEMEs.}
	\label{fig1}
\end{figure}

Under the influence of typical environmental stresses, the EEMEs are more susceptible to failure. Typical climatic conditions include  temperature (high temperature and low temperature), humidity, dry heat and low pressure, etc. \cite{IEC2003}. Particularly, to study the failure rate relationship under typical environmental stress, failure data from three typical regions of China are selected. These three regions are Xinjiang (XJ), Heilongjiang (HLJ) and Tibet (TI). 

The environmental characteristics of these three regions are summarized as Table \ref{table1}. It shows that the XJ has the characteristics of dry heat. The  climate in HLJ is much colder. In contrast, TI has a lower air pressure. More importantly, these stresses are simultaneously superimposed on the outdoor EEMEs, which is more complicated than the experimental environment.

Using the AMI system, the failure rate data of smart electricity meters is given to analysis as shown in Fig. \ref{fig0}, where the data is randomly obtained from each typical environmental region. Due to regional differences, the sample sizes of XJ, HLJ and TI are 41, 7 and 10 respectively. In Fig. \ref{fig0}, the failure rate can be obtained by dividing the number of failures per year by the number of remaining smart electricity meters. 
The trend of failure rate varies from region to region. Meanwhile, it can be seen from Fig. \ref{fig0}(a)  that some of the points are sparse and deviate from dense areas, especially in 2015 and 2017. Intuitively, these data points can be considered as potential outliers.
Next, we use the WNB to study the effects of typical environmental stresses.

\begin{table}
	\caption{Environmental characteristics in typical regions}
	\label{table1}
	\setlength{\tabcolsep}{5pt}
	\renewcommand\arraystretch{1.1}
	\begin{center}
		%		\begin{tabular}{p{80pt} p{25pt} p{25pt} p{25pt} p{25pt} p{25pt}}
		\begin{tabular}{c c c c c}	
			\hline\hline \\[-3mm]
			\multicolumn{1}{c}{Typical} &  \multicolumn{4}{c}{Annual average environmental characteristics} \\ \cline{2-5}
			region & $ \text{Low Tem.(\textcelsius)} $ & $ \text{High Tem.(\textcelsius)} $  & $ \text{Hum.(\%RH)}$ & $ \text{Pre.(hPa)} $  \\
			\hline
			XJ   & -23      & 29       & 18.3            & 1003.5     \\
			HLJ  & -31      & 37       & 65.7             & 925.2     \\
			TI   & -10      & 29       & 81.9             & 605.2     \\
			\hline\hline 
		\end{tabular}
	\end{center}
\end{table}

\begin{figure*}
	\centering
	\includegraphics[width=0.97\textwidth]{Picture//rawdata.png}
	\caption{The failure rate of EEMEs in different regions. (a) XJ, (b) XZ, (c) TI.}
	\label{fig0}
\end{figure*}





\section{Weight Selection based on Hybrid-based Outlier Detection}
\label{Section3:knn}

Actual failure samples may introduce outliers during transmission. To tackle this problem, two common outlier processing methods are used include statistical and distance-based methods \cite{6684530}. Statistical method can determine outliers, however, can not get the degree of deviation of outliers. Distance based method is prone to identify sparse data points as outliers especially for small sample data. Therefore, we propose a HOD method based on the kNN and Chauvenet criteria to integrate the advantages of the two methods.

\subsection{Outlier Detection using Hybrid-based Method}
We denote the raw failure rate data as $ D_{r}=\{(t,  X_{i,j,t}, y_{i,j,t}  ), j = 1,2,3\} $, where $ t $ is the statistical time of EEME failure, $ y_{i,j,t} $ is the $ i $th failure rate value at $ t $th year in area $ j $. The $ X_{i,j,t} $ are three typical environmental stresses in Table \ref{table1}, where the temperature stress consists of the temperature difference between the highest and lowest temperatures.

The kNN is a distance based method. For a data point $ q(x_{q}, y_{q}) \in D_{r} $ and its $ k^{th} $ nearest neighbor $ q(x_{k}, y_{k}) $ from $ D_{r} $ , the distance between $ q $ and $ q(x_{k}, y_{k}) $ can be defined as
\begin{equation}\label{eq12Euclidean}
D_{k}(q) = \sqrt{(x_{q} - x_{k})^{2} - (y_{q} - y_{k})^{2}}
\end{equation}
where the Euclidean distance $ D_{k}(q) $ is selected as the evaluation criteria of kNN distance.

Then the implementation of kNN can be summarized as the following steps \cite{Ramaswamy}: (1) Given a $ k $ and the proportion of outliers $ m $, calculate the distance $ D_{k}(q) $ of each input data $ D_{r} $ and then rank the points based on their $ D_{k}(q) $. 
(2) Point $ q $ is identified as an outlier when no more than $ m-1 $ remaining data points have a larger distance value. (3) Let $ q_{m} $ denote the outliers, the score of $ q_{m} $ are obtained according to the maximum distance value $ D_{k}(q_{m}) $ of its $ k^{th} $ nearest neighbor.

In effect, the outlier score of kNN will be affected because the Euclidean distance considers all environmental stresses to be of equal importance to outliers. This condition ignores the correlation between failure rate and stresses.
For example, the pressure is usually stable, while temperature and humidity are constantly changing. Based on this knowledge, the pressure stress may have a lower weight.

Therefore, we provide a weight for different stresses in the Euclidean distance, which can be rewritten as
\begin{equation}\label{eq13Euclidean}
D_{k}^{w}(q) = \sqrt{[w_{d}(x_{q} - x_{k})]^{2} - (y_{q} - y_{k})^{2}}
\end{equation}

The weight $ w_{d} $ represents the degree of influence of stress on the failure rate. To determine $ w_{d} $, we introduce the Spearman Correlation Coefficient (SCC) to  Eq. (\ref{eq13Euclidean}) \cite{8374923}. For failure rate data $ D_{r} $ with a sample size of $ n $, the SCC is given by
\begin{equation}\label{eq14spearman}
w_{d} = 1 - \frac{6\sum{d_{i}^{2}}}{ n(n^{2}-1) }
\end{equation}
where $ d_{i} = rank(t,  X_{i,j,t} ) - rank(y_{i,j,t}) $ is the rank difference between the stresses and failure rate. The score of the outlier becomes $ D_{k}^{w}(q_{m}) $ in its $ k^{th} $ nearest neighbor. 

However, kNN only considers the distance relationship of data. Some data will be misidentified because the distance is affected by the sample size. Thus the statistical characteristics of the failure rate data need to be considered. 

To prevent kNN from misjudged sparse data as an outlier, a constraint method based on Chauvenet criteria is introduced, i.e., hybrid-based outlier detection. Chauvenet criteria is a statistical method used to determine outliers that cannot be measured repeatedly \cite{5976081}, which can be expressed as
\begin{equation}\label{eq15Chauvenet}
C_{y} = n_{t}p(\frac{ |y_{i,j,t} - \overline{y_{i,j,t}} |}{\sigma_{y_{i,j,t}}}\sigma)
\end{equation}
where $ n_{t} $ is the number of failure rate samples when time is $ t $, $ \overline{y_{i,j,t}} ~ \text{and} ~ \sigma_{y_{i,j,t}}$  are the mean and standard deviation of the  failure rate $ y_{i,j,t} $, $ p(a\sigma) $ is the probability value of the normal distribution at $ a\sigma $. Generally, the $ y_{i,j,t} $  is identified as an outlier if $ C_{y} $ is lower than 0.5.

Then the data point $ q(x_{q}, y_{q}) $ can be judged as an outlier utilizing the following method
\begin{equation}\label{eq16Hybrid}
Outlier(q(x_{q}, y_{q})) = \text{kNN}(q) \cap C_{y} < 0.5
\end{equation}
where $ \text{kNN}(q) $ denotes the detection result of kNN. 

Finally, a smaller weight is assigned to the failure rate if its kNN score is higher. For normal data, the weight $ w_{i,j,t} $ is set to 1. To establish the relationship between weight and score, we give the following empirical equation
\begin{equation}\label{eq17weight}
w_{i,j,t} = \begin{cases}
~~~~~~~~~~1  & \text{if} ~ y_{i,j,t} ~ \text{is normal data},   \\
\exp(\frac{-D_{k}^{w}(q_{m})}{5\max(D_{k}^{w}(q_{m}))} ) & \text{if}~ y_{i,j,t} ~ \text{is outlier}.
\end{cases}
\end{equation}
where $ \max(D_{k}^{w}(q_{m})) $ is the maximum of $ D_{k}^{w}(q_{m}) $. It shows that $ w_{i,j,t} $ is inversely proportional to the score, which is consistent with the expected result.


\subsection{Parameter selection of HOD}
Two parameters of HOD directly affect the accuracy of outlier detection includes $ k $ and $ m $. However, outlier detection belongs to an unsupervised process, it is impossible to know the true outlier. To select the appropriate parameters, we use two criterias to evaluate the HOD results, Silhouettes Coefficient (SC) and Davies-Bouldin Index (DBI) \cite{ROUSSEEUW198753}. 

Outlier detection can be considered as a process of clustering, where outliers are one cluster and the normal data is another cluster. SC is a measure of the degree of combination and resolution of clustering. It applies to models that do not know the ground truth outlier labels. The main principle of SC is based on the difference between the average distance of points in one cluster and the average distance of other clusters \cite{ROUSSEEUW198753}.  The value range of SC is [-1, 1], wherein a larger value indicates better results HOD.

To ensure reliable results, we adopt another criterion DBI for further evaluation. DBI is defined as the average compactness and similarity between the outlier and the normal failure rate data. It measures the average distance between the cluster data points and cluster centroids \cite{Halkidi}. Generally, a DBI value closer to zero indicates a better clustering partition.

In HOD, parameters $ k $ and $ m $ can be chosen as the trade off between SC and DBI.


\subsection{Parameter selection experiment}
Based on the previous discussion, the relationship between SC, DBI scores and parameters $ k $ and $ m $ is studied to select the appropriate HOD parameters.

The failure rate data in TI is taken as an example, the value of $ k $ ranges from 1 to 29 with a step size of 2, and the $ m $ ranges from 0.02 to 0.2 with a step size of 0.02. As is shown in Fig. \ref{fig3}, different parameters have different scores. In Fig. \ref{fig3}(a), the SC has the largest score when $ k \in (15, 30) $ and $ m \in (0.1, 0.18) $. The lowest DBI score at $ k=25 $ and $ m=0.12 $. Therefore, the  $ k $ and $ m $ can be set to 25 and 0.12 respectively as a compromise.

In this way, HOD parameters $ [ k, m ]$ in XJ and HLJ can be set to [20, 0.1] and [15, 0.08], respectively. 



\begin{figure}
	\centerline{\includegraphics[width=0.49\textwidth]{Picture//KNN.png}}
	\caption{HOD parameter analysis based on SC and DBI. (a) Relationship between SC score and parameters $ k $ and $ m $, (b) Relationship between DBI score and parameters $ k $ and $ m $.}
	\label{fig3}
\end{figure}




\section{Failure rate analysis using WNB}
\label{Section2:bayes}

\subsection{Principle of Weighted Nonlinear Bayesian}
The next challenge is to evaluate failure rate. Compared with the linear Bayesian model, nonlinear Bayesian (NB) has a stronger fitting ability.  
The NB is expected to be more suitable for fitting data containing outliers.
%Meanwhile, the failure rate and life data are affected by the external environment and therefore have randomness and outliers.


Built upon the above HOD, we assign the weight $ w_{i,j,t} $ to each failure rate sample. For the failure rate model $ \Theta(X_{i,j,t}, t) $, we have the following equation
\begin{equation}\label{eq1}
w_{i,j,t}y_{i,j,t}  = \textbf{W} \Theta(X_{i,j,t}, t) 
\end{equation}
where $ w_{i,j,t} \in \textbf{W} $. Then the raw failure rate data can be written as $ D_{r}^{w}=\{(t,  X_{i,j,t},  w_{i,j,t}y_{i,j,t}  )\} $.

In general, for electronic instruments, the failure rate change interval $ \Delta{y} = w_{i,j,t}y_{i,j,t} - w_{i,j,t-1}y_{i,j,t-1}  $ obeys the Weibull distribution \cite{LIU201839} $ \Delta{y} \sim Weibull(\gamma, \beta) $. The probability density function of Weibull distribution is
\begin{equation}\label{eq2_weibull}
f(\Delta{y} | \gamma, \beta) = \gamma \frac{\Delta{y}^{\gamma - 1}}{\beta^{\gamma}}\exp[ -(\frac{\Delta{y}}{\beta})^{\gamma} ]
\end{equation}
where $ \gamma>0 ~ \text{and} ~ \beta >0 $ denotes the shape parameter and scale parameter of Weibull distribution, respectively. The shape parameter $ \gamma $  represents the failure mechanism of different EEMEs. The scale parameter $ \beta $ controls the failure distribution at a specific time. Thus, the shape parameter is usually subject to a fixed distribution. 



Unlike general linear regression, where the relationship between stresses and failure rate is linear. To establish a reasonable connection, a nonlinearly contact function is introduced to scale parameter $ \beta $. Utilizing the generalized Eyring model \cite{7430344}, a weighted nonlinear Bayesian method is proposed, in which the nonlinearly contact function can be expressed as
\begin{equation}\label{eq3Eyring}
\log(\beta) = \mu_{0} + \mu_{t1,j}t + \mu_{t2,j}t^{2} + \varPhi(x) + \varepsilon_{j}
\end{equation}
where $ \mu_{0}$ denotes the intercept, $ \mu_{ti,j}$ denotes the slope of time stress, $ log(\cdot) $ is used to constrain the value range of $ \beta $, $ \varepsilon_{j} $ denotes the error term to account for the discrepancy between the $ D_{r} $ and predicted value. $ \varPhi(x) $ denotes the generalized Eyring model and its form is 
\begin{equation}\label{eq4Eyring}
\varPhi(x) = \mu_{x0}\exp(\frac{\mu_{x1}}{T})\exp(\frac{\mu_{x2}}{P})\frac{\mu_{x3}}{(RH)^{\mu_{x4}}}
\end{equation}
where $ \mu_{xi} $ is the coefficient of environmental stress, $ T, P ~\text{and}~ RH $ are temperature, pressure and humidity stresses, respectively. To ensure model convergence, WNB model parameters require a suitable prior distribution.

\subsection{WNB Model Prior and Parameter Estimationon}
Hierarchical Bayesian structure is a powerful structure that allows multiple sub-group data to be fused together \cite{MISHRA201825}. Therefore, we introduce a two-levevl hierarchical modeling to model parameters.

Three methods are used to determine the prior distribution of model parameters, including range of values, noinformation prior distribution and weakly information prior distribution. 

In the first level of WNB, a positive distribution is assigned to $ \gamma $ according the Eq. (\ref{eq2_weibull}). The Half-Cauchy distribution is selected as the priori of  $ \gamma $, namely $ \gamma \sim Half-Cauchy(10) $. For the error term $ \varepsilon_{j} $, a normal distribution of smaller standard deviations is selected, $ \varepsilon_{j} \sim N(0, 1) $.
There is no additional value limit for the coefficients of time stress and environmental stress. Naturally, the weakly information prior distribution is obtained as follows
\begin{equation}\label{eq5prior}
\mu_{ti,j}, \mu_{xi}  \sim  N(u_{i}, {\sigma_{i}}^{2})
\end{equation}
where $ u ~\text{and}~ \sigma > 0 $ are the mean and standard deviation of Normal distribution, respectively.

In the second level of WNB, weakly information priors could be assigned to reduce the impact of priors on parameter updates when there is no additional experience reference\cite{7974778}. The conjugate prior distribution of mean $ u_{i} $ is normal distribution, so we further assumed the $u_{i} ~\text{and} ~\sigma_{i} $ as
\begin{equation}\label{eq6prior}
\begin{aligned}
u_{i} & \sim  N(0, \sigma^{2}) \\
\sigma_{i} & \sim  Half-Cauchy(\nu)
\end{aligned}
\end{equation}
where $ \nu $ is the scale parameter. To approximate no prior knowledge, a large value can be set to $ \sigma ~ \text{and} ~ \nu $ so that the parameters will be approximated as a flat hyperplane space.

To obtain the posterior distribution of WNB parameters, the joint posterior distribution of the Bayesian model is first calculated. According to the Bayes' theorem, the joint posterior distribution can be continuously updated based on observation data $ D_{r}^{w} $, which can be computed as
\begin{equation}\label{eq7posterior}
\begin{aligned}
p( \gamma, \mu_{0} , \mu_{ti,j}, \mu_{xi}, \varepsilon_{j}| D_{r}^{w} )  & =  \frac{p(\gamma, \mu_{0} , \mu_{ti,j}, \mu_{xi}, \varepsilon_{j}, D_{r}^{w})}{p(D_{r}^{w})} \\
& \propto p(D_{r}^{w}| \gamma, \beta ) p(\mu_{ti,j},\mu_{xi}| u_{i}, \sigma_{i}) \\
& \times p(\gamma) p( \mu_{0} ) p(u_{i} ) p(\sigma_{i})
\end{aligned}
\end{equation}
where $ p(D_{r}^{w}| \gamma, \beta ) $ is the likelihood, $ \beta $ consists of \{ $ \mu_{0} , \mu_{ti,j}, \mu_{xi}, \varepsilon_{j} $\}, $ p(D_{r}^{w}) $ is the  marginal density distribution, and it can be derived as
\begin{equation}\label{eq8posterior}
\begin{aligned}
p(D_{r}^{w}) = \iint p(D_{r}^{w}| \gamma, \beta ) p(\gamma, \beta)d\gamma d\beta
\end{aligned}
\end{equation}

Then each parameter can be evaluated according its marginal posterior density. To estimate $ \mu_{ti,j}  $, for example, the marginal posterior density of  $ \mu_{ti,j}  $ is evaluated using integral as follows:
\begin{equation}\label{eq9posterior}
p(\mu_{ti,j} |D_{r}^{w}) = \iiiint p( \gamma, \beta| D_{r}^{w} ) d\gamma d\mu_{0} d\mu_{xi} d\varepsilon_{j}
\end{equation}

The mean of each parameter can be further evaluated by integrating. More importantly, the reliability $ p_r $ of EEMEs at different stages can be obtained as
\begin{equation}\label{eq10reliabil}
p_r(t|D_{r}^{w}) = \exp [-\iint_0^t f(\Delta{y} | \gamma, \beta) d\gamma d\beta ]
\end{equation}

The value range of reliability $ p_r(\cdot) $ is [0, 1]. To predict the failure rate under different environmental stresses, the distribution of predicted failure rate can be obtained by combining the joint posterior probability. For new observed data $ D_{r}^{n} $, the predicted value can be expressed as
\begin{equation}\label{eq11reliabil}
p_n(D_{r}^{n}|D_{r}^{w}) = \iint E(D_{r}^{n}|\gamma,\beta) p( \gamma, \beta | D_{r}^{w}) d\gamma d\beta
\end{equation}
where $ E(D_{r}^{n}|\gamma,\beta) $ is expected value at a given $ \gamma,\beta $ value, $ p( \gamma, \beta | D_{r}^{w}) $ is the posterior distribution in Eq. (\ref{eq7posterior}).



\subsection{Nonlinear Model Experiment}

To verify the validity of the proposed nonlinear expression, we compare the Eq. (\ref{eq3Eyring}) with some commonly used regression analysis include Multivariate Linear Regression (MLR) \cite{FUMO2015332}, Nonlinear Regression with fixed effect intercept and slope (NRFIS) \cite{GUO2015173}. The detailed expression of these models are:
\begin{enumerate}
	\item MLR \cite{FUMO2015332}: MLP is a generalized linear model in which the variables are linearly superimposed. The MLP can be written as
	\begin{equation}\label{eq3MLP}
	\log(\beta) = \mu_{0} + \mu_{1}t + \mu_{2}T+\mu_{3}P+\mu_{4}RH + \varepsilon
	\end{equation}
	where $ \mu_{0}, \mu_{1}, ... \mu_{4} $ are the regression coefficients, $ \varepsilon $ is the error of the observation.
	\item MLR + Eyring model \cite{FUMO2015332}: To verify the role of the Eyring model, we transform the linear form of environmental stress into the Eyring model. The expression is
	\begin{equation}\label{eq3MLPEyring}
	\log(\beta) = \mu_{0} + \mu_{1}t + \varPhi(x) + \varepsilon
	\end{equation}
	\item NRFIS \cite{GUO2015173}: The intercept and slop are fixed compared to random effects. The time stress adopts nonlinear regression, and the environmental stresses adopt Eyring model, which can be expressed as
	\begin{equation}\label{eq3NRFIS}
	\log(\beta) = \mu_{0} + \mu_{1}\log{t} + \mu_{2}\sqrt{t} + \mu_{3}\sqrt{t}\log{t} + \varPhi(x) + \varepsilon
	\end{equation}
\end{enumerate}


The Root Mean Square Difference (RMSE), Mean Absolute Error (MAE) and Widely Applicable Information Criterion (WAIC) of different methods are listed in Table \ref{tabnoliear}. It shows that the MLR with Eyring has lower RMSE and WAIC values than MLR. This means that the Eyring model can describe the effects of environmental stresses more accurately. Meanwhile, compared with MLR and Eyring, the NRFIS has lower RMSE and MAE but has a higher WAIC value, indicating that NRFIS is not suitable for describing time stress with fixed parameters. The proposed method has the lowest value on each index, indicating the validity of the model.

\begin{table}
	\caption{Performance comparison of different regression methods}
	\label{tabnoliear}
	\setlength{\tabcolsep}{7pt}
	\renewcommand\arraystretch{1.2}
	\begin{center}
		\begin{tabular}{c c c c}	
			\hline\hline \\[-3mm]
			\multicolumn{1}{c}{Model} &  \multicolumn{1}{c}{RMSE} &  \multicolumn{1}{c}{MAE} &  \multicolumn{1}{c}{WAIC} \\ 
			\hline
			MLR    						& $ 9.5\times10^{-3} $  & $ 8.7\times10^{-3} $       & 578.35              \\
			MLR + Eyring    			& $ 8.3\times10^{-3} $  & $ 6.6\times10^{-3} $       & 522.54              \\	
			NRFIS    					& $ 7.8\times10^{-3} $  & $ 6.4\times10^{-3} $       & 526.43               \\
			\textbf{Proposed model}	& $ \bm{7.2\times10^{-3}} $ & $ \bm{5.3\times10^{-3}} $  & \textbf{463.57}               \\
			\hline\hline
		\end{tabular}
	\end{center}
\end{table}





\section{Failure Rate Evaluation Framework}
Using the proposed WNB and HOD, this section proposes a framework called WNB-HOD for failure rate prediction and evaluation. Its flowchart is shown Fig. \ref{fig2}, which the framework can be divided into the following two parts:
\begin{enumerate}
	\item Outlier detection: Performing outlier detection on the failure rate data $ D_{r} $ based on HOD, and then obtaining the weight $ w_{i,j,t} $ of the failure rate data.
	\item Failure rate evaluation of EEME: Establish a WNB model based on time and environmental stresses, and compare the effects of reliability and environmental stress.
\end{enumerate}

In the training process of the WNB model, the posterior distribution of the model parameters are estimated using No-U-Turn Sampler (NUTS), which is a Markov Chain Monte Carlo (MCMC) sampling method \cite{PyMC333}. To verify the performance of the model training, the WAIC and Leave-one-out Cross-validation (LOOC) are used to evaluate and compare the out-of-sample prediction accuracy.
Thereafter, the proposed WNB-HOD framework is further verified by different experiments.


\begin{figure}
	\centerline{\includegraphics[width=0.45\textwidth]{Picture//Framework.png}}
	\caption{The failure rate evaluation framework based on WNB-HOD.}
	\label{fig2}
\end{figure}






\section{Experiments}
\label{Section4}
To verify the effectiveness of the proposed WNB-HOD, many experiments are validated based on smart electricity meter data according to Seciton \ref{Section1:data}. Cross-validation method is used in the model training process. Eighty percent of the sample data for each region is used for training, and the remaining data is used for prediction. The simulation environment is Pymc3, which is a software library based on probabilistic programming. We set the total number of samples to 10000, of which 3000 are burn-in in MCMC.

\subsection{Model Convergence Verification}
According to Fig. \ref{fig2},  the SCC is first calculated as listed in Table \ref{tabSCC}. It shows that the time stress is most related to failure rate, while pressure is the least relevant. After calculating the SCC, the weights of different stresses can be specified when calculating the Euclidean distance according to Eq. (\ref{eq13Euclidean}).
\begin{table}
	\caption{The SCC between failure rate and different stresses}
	\label{tabSCC}
	\setlength{\tabcolsep}{7pt}
	\renewcommand\arraystretch{1.1}
	\begin{center}
		\begin{tabular}{c c c c c}	
			\hline\hline \\[-3mm]
%			\multirow{1}*{ } &  \multicolumn{4}{c}{Time and environmental stresses} \\ \cline{2-5}
			\multicolumn{1}{c}{} &  \multicolumn{1}{c}{Time.(t)} &  \multicolumn{1}{c}{Tem.(T)} &  \multicolumn{1}{c}{Hum.(RH)}  &  \multicolumn{1}{c}{Pre.(P)} \\ 
%				 & $ \text{Time.(t)} $ & $ \text{Tem.(T)} $  & $ \text{Hum.(RH)}$ & $ \text{Pre.(P)} $  \\
			\hline
			SCC    & 0.89      & 0.53       & 0.51             & 0.46      \\
			\hline\hline
		\end{tabular}
	\end{center}
\end{table}

To verify the convergence of WNB-HOD, the Geweke z-scores is utilized to test the sampling process of MCMC. The sampling trace is divided into a number of segments to obtain the difference. If a Bayesian model is convergent, the z-scores of MCMC sample trace will be between -1 and 1.

Fig. \ref{fig3Geweke} shows the Geweke z-scores of some WNB-HOD parameters. These parameters are randomly selected from different levels of WNB. As clear from Fig. \ref{fig3Geweke}, all the parameters are located between -1 and 1. This means that the WNB-HOD has satisfactory convergence.
\begin{figure}
	\centerline{\includegraphics[width=0.43\textwidth]{Picture//zscore.png}}
	\caption{The convergence verification of WNB-HOD using Geweke z-scores, where the first 10\% and the last 50\% of the MCMC sample trace are used for verification.}
	\label{fig3Geweke}
\end{figure}


\subsection{Performance Comparison With Common Methods}
To verify the predicted performance of WNB-HOD, we compare the proposed method with some other methods includs ANN \cite{KUTYLOWSKA201541}, Bayesian Exponential Model (BEM) \cite{1046900} and WNB without outlier detection. A three-layer ANN is used for analysis, where the number of nodes is optimally selected to be 20. The activation function is ReLU, the adaptive learning rate is used which the learning rate is reduced by half if the loss of consecutive cycles is no longer reduced.

The predicted performance of different models is shown in the Fig. \ref{fig4Comp}. It should be notable that the blue points denote the weighted data after HOD, and the gray points denote the raw failure data. It demonstrates that some of the sparse data with high failure rate are judged as outliers. The weighted data is closer to the prediction curve so that the effect of the outliers can be reduced.
Obviously, BEM does not change with failure rate, especially after 2017. In Fig. \ref{fig4Comp}(a) and (c), the prediction result of ANN approximates a straight line, indicating that the effect of stress on failure rate has not been learned. 
Compared with WNB, NB has larger predicted values in 2016-2017 of Fig. \ref{fig4Comp}(a) and 2018 of Fig. \ref{fig4Comp}(c). It means that NB is more susceptible to outliers. The result shows that WNB-HOD can accurately predict failure rate and is insensitive to outliers.


\begin{figure*}
	\centering
	\includegraphics[width=0.97\textwidth]{Picture//Pred1.png}
	\caption{Comparison of failure rate prediction for four methods. (a) XJ, (b) XZ, (c) TI.}
	\label{fig4Comp}
\end{figure*}


To quantify the fitting error, different methods of RMSE and MAE are used. On the other hand, the Average Confidence Interval Width (ACIW) is used to determine the confidence intervals for probabilistic models, which can be defined as
\begin{equation}\label{eq21ACIW}
\text{ACIW} = \frac{1}{n}\sum_{i=1}^{n} [p_n(D_{r}^{n}|D_{r}^{w})_{97.5} - p_n(D_{r}^{n}|D_{r}^{w})_{2.5}]
\end{equation}
where $ p_n(D_{r}^{n}|D_{r}^{w})_{97.5} ~ \text{and}~ p_n(D_{r}^{n}|D_{r}^{w})_{2.5} $ denote the 97.5 and 2.5 quantiles of the failure rate prediction distribution.


The comparison of prediction errors are listed in Table \ref{tab4SCC}, where the errors in different regions are calculated separately. It shows that both the RMSE and  MAE of ANN are higher than BEM and NB. WNB-HOD has a smaller prediction error compared with NB, indicating that HOD can reduce the fitting error. 
Moreover, the Confidence Interval (CI) indicates the degree of certainty of the fitting result, and a narrower CI means more certain results. Experimental results show that WNB has a narrower ACIW. This means that the uncertainty of failure rate prediction is smaller. More importantly, it shows that ANN cannot provide CI information.
In terms of overall effect, WNB-HOD is more suitable for small sample data models with noise interference.

\begin{table}
	\caption{Comparison of prediction errors of different methods}
	\label{tab4SCC}
	\setlength{\tabcolsep}{5pt}
	\renewcommand\arraystretch{1.2}
	\begin{center}
		\begin{tabular}{c c c c c}	
			\hline\hline \\[-3mm]
			\multicolumn{1}{c}{Index} &  \multicolumn{1}{c}{Model} &  \multicolumn{1}{c}{XJ}&  \multicolumn{1}{c}{HLJ}&  \multicolumn{1}{c}{TI}\\
			\hline
			\multirow{4}*{MAE} 		&  ANN   		& $ 1.0\times10^{-2} $       & $ 1.2\times10^{-2} $      & $ 9.2\times10^{-3} $  \\
			    					& BEM        	& $ 6.1\times10^{-3} $       & $ 7.6\times10^{-3} $      & $ 4.2\times10^{-3} $  \\
			    					& NB        	& $ 5.6\times10^{-3} $       & $ 7.4\times10^{-3} $      & $ 4.3\times10^{-3} $  \\
			    				& \textbf{WNB-HOD}  & $\bm{5.0\times10^{-3}}$    & $\bm{7.1\times10^{-3}} $  & $ \bm{3.9\times10^{-3}} $  \\  \cline{2-5}
			\multirow{4}*{RMSE} 	&  ANN   		& $ 1.2\times10^{-2} $       & $ 1.5\times10^{-2} $      & $ 1.0\times10^{-2} $  \\
									& BEM        	& $ 8.7\times10^{-3} $       & $ 1.1\times10^{-2} $      & $ 5.6\times10^{-3} $  \\
									& NB        	& $ 8.1\times10^{-3} $       & $ 1.0\times10^{-2} $      & $ 5.7\times10^{-3} $  \\
								& \textbf{WNB-HOD} & $ \bm{6.9\times10^{-3}} $    & $ \bm{9.7\times10^{-3}} $ & $ \bm{5.0\times10^{-3}}$  \\  \cline{2-5}
			\multirow{3}*{ACIW} 	&  BEM   		& 3.68   	 & 3.12           & 2.33  	\\
									&  NB   		& 2.93   	 & 3.20           & 2.55  	\\
									& \textbf{WNB-HOD}  & \textbf{2.59}     & \textbf{2.94}  & \textbf{2.29}      \\
			\hline\hline
		\end{tabular}
	\end{center}
\end{table}

Furthermore, the information criteria results are listed in Tabel \ref{tab5SCC}. The dWAIC and dLOOC represent the number of effective parameters of the model. The $ w $ is the weight of the model, indicating the ability of the model to fit the failure rate.
It can be seen that WNB-HOD has a lower WAIC and LOOC value. Correspondingly, the dWAIC and dLOOC values are lower compared with BEM. It means that BEM requires more parameters to fit the failure data. A higher $ w $ implies a better performance of the WNB-HOD model.

\begin{table}
	\caption{Comparison of information criteria in Bayesian model}
	\label{tab5SCC}
	\setlength{\tabcolsep}{7pt}
	\renewcommand\arraystretch{1.1}
	\begin{center}
		\begin{tabular}{c c c c c}	
			\hline\hline \\[-3mm]
			\multicolumn{1}{c}{Information} &  \multirow{2}*{Model} &  \multicolumn{1}{c}{WAIC/} &  \multicolumn{1}{c}{dWAIC/} &  \multirow{2}*{$ w $}\\
			\multicolumn{1}{c}{Criterion} &							&  LOOC			&  dLOOC	  		\\
			\hline
			\multirow{2}*{WAIC} 		&  BEM   		& 642.04   	 & 47.22            & 0 		\\
										&  NB   		& 520.70   	 & 7.75             & 0.08 		\\
							& \textbf{WNB-HOD}  & \textbf{463.57}    & \textbf{7.33}    & \textbf{0.92}      \\  
			\multirow{2}*{LOOC} 		&  BEM   		& 632.14   	 & 42.26            & 0  	\\
										&  NB   		& 520.91   	 & 7.86             & 0.08 		\\
							& \textbf{WNB-HOD}    & \textbf{461.79}  & \textbf{7.84}    & \textbf{0.92}      \\
			\hline\hline
		\end{tabular}
	\end{center}
\end{table}



\subsection{Parameter Interpretation and Reliability}
Bayesian model is interpretable compared to ANN. The posterior parameter values of time and environmental stresses are listed in Table \ref{tab6parameter}. All the $ \mu_{t1,j} $ are positive, indicating that the time stress is positively correlated with the failure rate. It shows that the growth rate of failure rate in TI is the biggest because $ \mu_{t1,2} $ is the largest. For environmental stress, the coefficient $ \mu_{x1} $ of temperature is the largest, indicating that the humidity has a greater impact on EEMEs. The $ \mu_{x3} $, coefficient of humidity, and its narrower CI indicates that the effect of humidity on the failure rate is more certain.

\begin{table}
	\caption{Time and environmental stress parameter values of WNB}
	\label{tab6parameter}
	\setlength{\tabcolsep}{7pt}
	\renewcommand\arraystretch{1.2}
	\begin{center}
		\begin{tabular}{c c c c c}	
			\hline\hline \\[-3mm]
			\multirow{2}*{Parameters} & \multirow{2}*{Mean} &  \multicolumn{2}{c}{CI} &  \multicolumn{1}{c}{Standard} \\  \cline{3-4}
			 &   &   $\text{2.5\%}$  & $ \text{97.5\%}$ &   \text{deviation} \\
			\hline
			$ \mu_{t1,0} $    	& 1.24     & 1.07         & 1.45      & 0.09             \\
			$ \mu_{t1,1} $    	& 1.26     & 1.05         & 1.47      & 0.11             \\
			$ \mu_{t1,2} $    	& 1.27     & 1.08         & 1.52      & 0.12             \\
			$ \mu_{x1} $    	& 11.56    & -80.96       & 108.49    & 49.295           \\
			$ \mu_{x2} $    	& 1.22     & -81.73       & 131.58    & 54.38            \\
			$ \mu_{x3} $    	& 2.10     & 1.37         & 2.98      & 0.48             \\
			\hline\hline
		\end{tabular}
	\end{center}
\end{table}

The reliability curve of smart electricity meter is shown in the Fig. \ref{fig6Reliability}. It shows that the reliability decreases with time in the typical environment. Moreover, the predicted reliability curve is close to the actual reliability curve, indicating that the result is accurate.
Particularly, the rate of reliability degradation increases in the later stages of equipment operation. In the next few years, the density function of reliability is more widely distributed, and uncertainty begins to increase.
 Therefore, consumers of smart electricity meters should take more equipment maintenance and storage measures.
\begin{figure}
	\centerline{\includegraphics[width=0.43\textwidth]{Picture//Reliabilityweibull.png}}
	\caption{Reliability of smart electricity meter based on WNB-HOD.}
	\label{fig6Reliability}
\end{figure}


\section{Conclusion}
\label{Section5}
In this paper, a weighted nonlinear Bayesian and hybrid-based outlier detection were proposed for the failure rate estimation of EEME. The HOD method consists of kNN and Chauvenet criteria, which combines the characteristics of distance-based and statistical methods. The fitting error and CI experiments verified the effectiveness of HOD. The weight generated by the HOD was assigned to the failure data, and then a WNB was proposed to fuse the weight data from different regions. Different regression equations experiments showed that WNB has a lower prediction error. Particularly, the case study of electric energy meters was used to verify the WNB-HOD framework. The comparison experiments of different methods showed that the WNB-HOD has strong outlier detection ability and fitting performance for small samples. Finally, the relationship between different time and environmental stresses and failure rates was obtained and evaluated. Reliability prediction results can provide a strategy for practical EEME reliability regulation and management. 



\bibliographystyle{IEEEtran}
\bibliography{QWbib}

\end{document}
